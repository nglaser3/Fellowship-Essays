\documentclass[12pt]{letter}
\usepackage{graphicx}
\usepackage[letterpaper,top=2.54cm,bottom=2.54cm,left=2.54cm,right=2.54cm]{geometry}
\usepackage{mathptmx} %times-new-roman

\begin{document}

As part of my research experience, I investigated the applicability of Functional Expansion Tallies (FETs) to complex, non-homogenous reactor geometries via high-fidelity simulation with OpenMC, a Monte-Carlo neutron transport code. The most complex of these geometries was a TRistructural ISOtropic fuel (TRISO) particle compact --- a small hexagonal graphite block filled with tens of thousands of TRISO particles. I ran this simulation with 500,000 particles and 1000 batches (500 inactive and active). Further, the FET was a zernike expansion, and was expanded to the 200th radial order (20,101 polynomials). This FET was tallying kappa-fission (heat generated by a fission event, related to power), and was evaluated for each fission event that occured (this is the definition of a FET but like idk i feel like this emphasizes how hard this was). I ran this problem on 4 CPU cores, and the wall time of this simulation was roughly four hours. From this simulation, I determined that a 200th order zernike FET was inadequate to approximate the radial and angular power distribution of the compact due to extremely steep power gradients at material interfaces. 

If I had 100 times more powerful computational resources, I could have expanded the same FET to a much higher radial order, potentially finding an order that adequately approximates the steep power gradients present. Alternatively, I could have drastically increased the size of my model, introducing many more compacts and expanding over a full array and potentially reducing the affects of hyper-local power dependences on the expansion, better approximating the mean distribution. However, this array of hexagonal compacts results in mathematical difficulties. Investigating a single compact allows for perfect geometric representation with the zernike family, for the domain of orthogonality is the unit disc, and a hexagonal compact has a cylindrical center that is filled with the TRISO particles where the kappa-fission is non-zero. A full array of hexagonal compacts would result in a non-perfect circle, and would require a sophisticated mapping of a unit disc to the n-sided polygon --- something not yet available in OpenMC.

\end{document}