\documentclass[12pt]{letter}
\usepackage{graphicx}
\usepackage[letterpaper,top=2.54cm,bottom=2.54cm,left=2.54cm,right=2.54cm]{geometry}
\usepackage{mathptmx} %times-new-roman

\begin{document}

An important challenge currently facing the field of nuclear energy is the deployment of next generation nuclear reactors. Advanced reactors are extremely desirable because they offer greater thermal efficiency, improved fuel cycles, and improved safety compared to the currently deployed Light Water Reactors (LWRs). The industry was able to rely on precedent and vast experimental data to support the licensing and deployment of the current 'fleet' of LWRs. Unfortunately, there is little precedent or experimental data available today to support advanced reactor deployment. High-fidelity simulations can be utilized to adress this, filling in the gaps caused by a lack of reference data. Further, advanced reactors typically have additional, more complicated physics that need to be simulated with high accuracy to be useful, thus high spatial- and temporal-resolution models are required for meaningful analysis. 

There are various existing computational tools for modeling the neutron population with high spatial and temporal resolution (Moltres, OpenSN, etc.), but few operate efficiently on the new HPCs leveraging GPU architectures. I aim to adress this, developing methods to adapt deterministic neutron transport methods to GPU architectures. There are two methods I am particularly interested in --- domain decomposition and The Random Ray Method (TRRM) for the Method Of Characteristics (MOC). Domain decomposition has been proven with success on CPU architectures, and is the dominant method utilized for multi-processor simulations for determinstic neutron transport. Further, domain decomposition has been proven to work for GPU architectures with Shift, a stochastic neutron transport code. Thus, domain decomposition used for GPU acceleration of deterministic neutron transport is a promising route forward. Second, TRRM is a version of MOC that is still trivially parallelizable, but does not suffer the severe memory constraints present in MOC. TRRM, retaining its minimal memory demand, has not been adapted to GPU architectures, as utilizing the same parallelization scheme as CPUs for GPUs does not saturate the GPUs with adequate work, and is thus inneficient. An approach to this dilemma has been developed in the field of computer graphics --- Radiance Cascades (RC). RC adapts TRRM from long to short characteristics, forking at the termination of each yielding a new short characteristic for the GPU to solve over. This method not only saturates the GPUs in use, but also greatly increases the spatial-resolution of the answer with the same number of starting locations as TRRM. This method has been developed for single-core 2-D problems, and I would like to pursue adapting this method to multi-core 3-D problems. To do this, I believe domain decomposition to again be a fruitful path to begin investigating.
If GPU acceleration of determinstic neutron transport is accomplished, the design, development, and licensing process of advanced reactors will be streamlined. With this, the deployment of next generation reactors can commence, transitioning the United States from predominantly carbon-emitting energy sources to nuclear, aiding in the realization of the United State's carbon emission goals.

\end{document}


