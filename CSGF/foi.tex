\documentclass[12pt]{letter}
\usepackage{graphicx}
\usepackage[letterpaper,top=2.54cm,bottom=2.54cm,left=2.54cm,right=2.54cm]{geometry}
\usepackage{mathptmx} %times-new-roman

\begin{document}

A key challenge in nuclear energy is enabling the deployment of next-generation
nuclear reactors. These reactors offer improved efficiency, enhanced fuel
cycles, and greater safety compared to Light Water Reactors (LWRs). While the
deployment of LWRs was supported by precedent and extensive experimental data,
advanced reactors lack this foundation. High-fidelity simulations can fill this
gap, enabling accurate modeling where data is scarce. These simulations are
particularly critical for advanced reactors, which require high spatial and
temporal resolution to capture their complex physics.

While computational tools exist for modeling neutron populations with high
spatial and temporal resolution, few are optimized to leverage modern
high-performance computing (HPC) systems with GPU architectures. My research
seeks to address this by adapting deterministic neutron transport methods for
acceleration with GPUs. I aim to explore two methods: domain decomposition and
The Random Ray Method (TRRM) for the Method of Characteristics (MOC).

Domain decomposition has been successfully applied on CPU architectures for
multi-processor simulations of deterministic neutron transport and for GPU
architectures in Shift, a stochastic neutron transport code. This makes it a
promising approach for deterministic simulations. TRRM, a variation of MOC,
reduces memory constraints while remaining parallelizable. However, traditional
CPU-based parallelization schemes for TRRM are inefficient on GPUs. A potential
solution lies in Radiance Cascades (RC), a method from computer graphics that
transforms TRRM's long characteristics into cascading series of short
characteristics. This technique not only saturates GPUs with sufficient work
but also improves spatial resolution. While RC have been demonstrated for
single-core, two-dimensional problems, I aim to extend it to multi-core,
three-dimensional applications.

Accelerating deterministic neutron transport with GPUs could expedite advanced
reactor design, development, and licensing. By streamlining these processes,
next-generation reactors could be deployed more rapidly, advancing the
transition to clean nuclear energy and helping achieve the United States’
carbon emission reduction goals.
\end{document}


